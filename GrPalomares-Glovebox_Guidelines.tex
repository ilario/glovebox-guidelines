\documentclass[a4paper,15pt]{scrartcl}
\usepackage[utf8]{inputenc}
\usepackage[english]{babel}
\usepackage{hyperref}
\hypersetup{unicode=true, pdftex, colorlinks=true, linkcolor=blue, citecolor=blue, filecolor=blue, urlcolor=blue, pdftitle=Gr. Palomares - Glovebox Guidelines, pdfauthor=Ilario Gelmetti, pdfsubject=Glovebox}
\usepackage{geometry}
\geometry{
	top=1cm,
	inner=0.5cm,
	outer=0.5cm,
	bottom=1cm,
	headheight=1ex,
	headsep=1ex,
}
\usepackage{titling} % for moving up the title
\setlength{\droptitle}{-5em}% for moving up the title
\usepackage{changes} % for having \textsubscript
\pagenumbering{gobble}% Remove page numbers (and reset to 1)

\title{Glovebox TR1.14 Guidelines}
\subtitle{}
\author{ICIQ - Gr. Palomares - Ilario Gelmetti}
\date{}
\begin{document}

\maketitle

\section*{How to use the antichambers}
\begin{itemize}
	\item     Don't introduce in the antichamber closed \textbf{containers} with oxygen: insert them \textbf{opened or filled with inert gas}.
	\item     If you introduce closed containers, consider that the cap could pop out because of the vacuum and spread your compound in the antichamber, take proper countermeasures.
	\item     \textbf{Never introduce solvents containing water.}
\end{itemize}
\subsubsection*{Detailed procedure for inserting small objects:}
\begin{enumerate}
	\item     Make vacuum in the antichamber (this pulls the internal door in the proper position);
	\item     refill with nitrogen;
	\item     \textbf{leave the tap in closed position} (if you leave it in refill position acts like a big hole in the GB);
	\item     open the external door and introduce your objects, all the containers should be \textbf{opened}. \textit{E.g.} Petri dishes should be inserted opened;
	\item     close and make vacuum, wait until the vacuum gauge goes under its minimum and some time more;
	\item     refill \textbf{slowly} (the gas flow could throw your stuff around);
	\item     when the gauge approaches the max pressure switch again to vacuum (don't leave it at max pressure unless it's the last cycle);
	\item     repeat two more times (a total of \textbf{three vacuum-nitrogen cycles} for the small antichamber);
	\item     refill, open the internal door, remove your object, close the internal door and \textbf{leave the antichamber under vacuum} (doors are leaky).
\end{enumerate}

\subsubsection*{Detailed procedure for extracting small objects:}
\begin{enumerate}
	\item     \textbf{Make vacuum} in the antichamber (even if you think it's under nitrogen, it is not: doors of the small antichamber are leaky);
	\item     refill with nitrogen, open the internal door, insert your stuff, close the internal door taking care it goes in the right position;
	\item     make vacuum in the antichamber (this pulls the internal door in the proper position);
	\item     refill with nitrogen and \textbf{leave the tap in closed position} (if you leave it in refill position acts like a big hole in the GB);
	\item     open the external door and remove your objects;
	\item     close the door and \textbf{leave under vacuum}.
\end{enumerate}

\subsubsection*{Procedure for inserting big objects:}
\begin{itemize}
	\item     For using the big antichamber follow the same instructions as above but \textbf{six vacuum-nitrogen cycles} are suggested.
	\item     If possible, heat the big objects and introduce them while they're still hot.
	\item     Leave the big antichamber in \textbf{static vacuum}.
\end{itemize}


\section*{Before starting working}
\begin{itemize}
	\item     \textbf{Record the date, your name and H\textsubscript{2}O, O\textsubscript{2} concentrations} and all significant events (regenerations, malfunctions...) on the glovebox log book.
	\item     If you're going to use the spin coater or \textbf{solvents}, even if in small amounts, which damage the oxygen removal catalyst (methylene chloride, acetonitrile, alcohols, amines), \textbf{stop the circulation} purifier while working then, when you finish, make a \textbf{quick purge} (10-40 min, in the Functions menu of the GB controller) and switch on again the circulation purifier (nitrogen is much cheaper than a replacement for a damaged catalyst).
	\item     If you are using solvents whose vapours accumulation can damage your products, additionally keep the quick purge functioning for all the duration of your experiment.
	\item     When starting using the GB \textbf{lower the internal pressure} to approx 3-5 mbar (using the GB with too high pressure (14 mbar) makes the circulation purifier and the analyser to shutdown).
	\item    For using the \textbf{spin coater vacuum pump}: first switch on the small pump and then open the vacuum tap on the back of the GB (remember to close and switch off later).
	\item     If you notice that the oil level of the small pump is low, refill it. We're in charge of the small pump maintenance. 
	\item     Use \textbf{lab coat and clean gloves} (the GB gloves are not clean, at all), if you have a watch, bracelet or rings better to remove them (for reducing the risk of making holes).
	\item     Always use aluminium foil for keeping clean the internal part of the spin coater.
	\item     If you're going to work with solvents put \textbf{gloves on also in the internal side} of the GB (do it, they're not uncomfortable, indeed they tighten the black gloves on your fingers giving a better grip).
	\item     If the big or small antichamber pressure is not very low, report it to \textit{Chromatography, Thermal Analysis \& Electrochemistry Unit}: rubber rings need replacement.
\end{itemize}


\section*{After using the glovebox}
\begin{itemize}
	\item     \textbf{Label everything} and keep your products closed and \textbf{in a box with your name} on it, periodic cleaning will trash things happily \& arbitrarily, keep clean and you won't have to complain in vain.
	\item     \textbf{Throw the wastes} in a container and take it with you when you finish working, leave as little mess as possible for the next user, otherwise your wet wastes will go on contaminating the glovebox and poisoning the oxygen removal catalyst.
	\item     Leave the spin coater lid open, this way the droplets and residual solvent vapors can get removed by the purge process.
	\item     Leave the \textbf{small antichamber under static vacuum}.
	\item     Leave the big antichamber in static vacuum.
	\item     If you used the spin coater remember to first \textbf{close the vacuum tap} and then to \textbf{switch off the small pump}.
	\item    When finished using the GB \textbf{increase the internal pressure} to approx 10-12 mbar.
\end{itemize}


\section*{General guidelines}
\begin{itemize}
	\item     Remember that the small antichamber doors have leaks, vacuum-nitrogen cycle the antichamber even if it shouldn't be needed: \textbf{When in doubt - pump it out!} Do not make any assumptions!
	\item     Keep your samples open only when necessary. This is to avoid GB contamination and contamination of your samples.
	\item     When weighing materials in the glovebox, static electricity is a big problem. To minimize this disturbance use the antistatic gun and steel weighing boats.
	\item     Never keep needles around, throw them in the proper container, needles and cutting edges (\textit{e.g.} glass sharp edges, scissors) are gloves' worst enemies.
	\item     Don't use the antichamber vacuum for removing solvents, the pump has no solvents trap!
	\item     Enter and exit your arms in the glovebox \textbf{slowly} so that the overpressure doesn't get negative or too high (over 14 mbar, causing the GB circulation and analyser to shutdown).
\end{itemize}

\end{document}
